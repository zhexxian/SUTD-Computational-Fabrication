\documentclass{article}
\usepackage{fullpage}
\usepackage{graphicx}
\usepackage{hyperref}
\usepackage{color}
\usepackage{amssymb}
\usepackage{amsmath}
\usepackage{comment}
\usepackage{hyperref}
\usepackage{wrapfig}

\addtolength{\parskip}{\baselineskip}

\begin{document}

\section*{01.110: Computational Fabrication Summer 2017}

\subsection*{Lab 2: Basic 3D Printing}

\subsection*{May 30th (Tues) or Jun 1st (Thurs) 11:30am-1:30pm Fab Lab Ground Floor. Please check your respective slot.}

\subsection*{\textcolor{red}{Report Due Jun 11th at 11:55pm.}}

%\maketitle

\section{Overview}
In this lab, you will 3D print 1) a model of a scanned object captured in Lab 1, 2) a procedurally defined model from Assignment 2.

\section{Software Requirement}
This lab requires the use of the MankatiUM software. The software is available on Windows. It can be downloaded from \url{http://www.mankati.com/download/}. 

Check the "How our entry level 3D\-printer works\_v2.pdf" for instructions on using the MankatiUM software. The software portion is from "6-Steps to slice your model" onwards. 

You can also download and install the Meshmixer software by Autodesk. The software is useful for cleaning up the scanned meshes. The software is available at
\url{http://www.meshmixer.com/download.html}. 

\section{3D Printer Setup}
Please read the "How our entry level 3D\-printer works\_v2.pdf" in the zip file. The fab Lab technician will also help you during the lab. 

\section{Printing an OpenSCAD Object}

The process to print an OpenSCAD object can be outlined as follows:

\begin{enumerate}
%\item If your object requires support structures, compute the support in the Meshmixer. Save the model with the support structures.
\item Export the object with OpenSCAD as an .stl 3D model.
\item Load the model into the 3D printing software, select the appropriate settings, slice the model (generate the gcode), and save the sliced model file (.gcode) to a SD card. 
\item Prepare the 3D printer (check filament, nozzle, preheat, etc.).
\item Plug in the SD card into the printer and print the file.
\item After printing, remove the printed model from the build tray.

Please refer to the "How our entry level 3D\-printer works\_v2.pdf" for more details and the exact steps needed.
\end{enumerate}

\section{Printing a Scanned Object}

In general, printing the model of a scanned object should be similar to printing an OpenSCAD model.
However, you have to make sure that your model that is watertight and does not have many mesh defects.
You can fix your model by running the inspector in the Meshmixer ('Analysis' $\rightarrow$ 'Inspector') or by running the Poisson surface
reconstruction in the MeshLab. 

\section{Submission}

\noindent Submit your assignment on e-dimension by \textcolor{red}{\textbf{June 14th by 11:55pm}}. Please submit a single archive (\texttt{.zip} or
\texttt{.tar.gz}) containing:

\begin{itemize}
\item A write up for your group in either a text file or a pdf file. Describe any difficulties/experiences during the
3D printing.
\item Include some screenshots of your 3D prints ( the test object, the OpenSCAD object, and the
scanned object).
\item Write down the \textbf{names} of your group members. 
\end{itemize}
\end{document}
